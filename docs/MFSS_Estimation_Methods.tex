\documentclass[12pt]{article}
\usepackage{amsthm,amsfonts,amsmath,xcolor,fullpage}
\usepackage[letterpaper, margin=1in]{geometry}
\usepackage{enumitem}
\usepackage{parskip}

\setlength{\parindent}{0pt}

\newcommand{\Gt}{G_\theta}
\newcommand{\vecop}[0]{\text{vec}}
\newcommand{\Cov}[0]{\text{Cov}}
\newcommand{\Var}[0]{\text{Var}}
\newcommand{\flag}[0]{red}
\newcommand{\tr}[0]{\text{tr}}

\title{MFSS Estimation Methods}
\date{}
\author{}

\begin{document}

\maketitle

	We are interested in estimating state space models with time-varying parameters of the form
	\begin{align*} 
	y_t = Z_t \alpha_t + d_t + \varepsilon_t &\qquad \varepsilon_t \sim \mathcal{N}(0,H_t) \\ 
	\alpha_{t} = T_t \alpha_{t-1} + c_t + R_t \eta_t &\qquad \eta_t \sim \mathcal{N}(0, Q_t) \qquad
	\alpha_1 \sim \mathcal{N}(a_1, P_1) 
	\end{align*} 
	The elements of the parameter matrices $Z_t$, $d_t$, $H_t$, $T_t$, $c_t$, $R_t$, and $Q_t$ are divided into those that are known and those that are to be estimated as a function of the parameter vector $\theta$. We are interested in the maximum likelihood estimate of those parameters given data $Y_n = \{y_1, \dots, y_n\}$. 

	Instructions for how to set up the state space and estimate the maximum likelihood of the unknown parameters can be found in the MFSS user guide. This document attempts to detail how the likelihood and its gradient are calculated for a given system, and how the parameters in $\theta$ will be estimated.

\section{Kalman Filtering and Smoothing Calculations}

\subsection*{Multivariate Filter}
	As shown in \cite{dk_book} (\S 4.3), the filtered estimates of the state, $a_t = \mathbb{E}(\alpha_t | Y_t)$ and $P_t = \Var(\alpha_t | Y_t)$, are given by 
	\begin{align*}
	a_{t+1} = T_{t+1} a_t + c_{t+1} + K_t v_t &\qquad
	P_{t+1} = T_{t+1} P_t L_t^\prime + R_{t+1} Q_{t+1} R_{t+1}^\prime \\
	v_t = y_t - Z_ta_t - d_t &\qquad
	K_t = T_{t+1} P_t Z_t^\prime F_t^{-1} \\
	F_t = Z_t P_t Z_t^\prime + H_t &\qquad
	L_t = T_{t+1} - K_t Z_t 
	\end{align*} 

	and the smoothed estimates, $\hat{\alpha}_t = \mathbb{E}(\alpha_t | Y_n)$ and $V_t = \Var(\alpha_t | Y_n)$ are given by
	\begin{align*}
	\hat{\alpha}_t = a_t + P_t r_t &\qquad V_t = P_t - P_t N_t P_t \\ 
	r_t = Z_t^\prime F_t^{-1} v_t + L_t^\prime r_{t+1} &\qquad N_{t} = Z_t^\prime F_t^{-1} Z_t + L_t^\prime N_{t+1} L_t
	\end{align*}
	where $r_{n+1} = 0$ and $N_{n+1} = 0$. 

\subsection*{Univariate Filter}
	Substantial computational gains are available using the univariate Kalman filter by avoiding the inversion of the $F_t$ matrix above (as well as enabling the use of the exact initial filter, see below). For more details, see \cite{dk_book} (\S 6.4). 

	When any $H_t$ is non-diagonal, the observation equation is transformed by taking the LDL factorization of the $H_t$ matrices. The transformed parameters are marked with $^*$ to denote that they have been transformed.
	\begin{align*}
	y_t^* = Z_t^* \alpha_t + d_t^* + \varepsilon_t^* &\qquad \varepsilon_t^* \sim N(0, H_t^*)\\
	y_t^* = C_t^{-1} y_t \qquad Z_t^* = C_t^{-1} Z_t \qquad d_t^* &= C_t^{-1} d_t \qquad \varepsilon_t^* = C_t^{-1} \varepsilon_t \qquad H_t = C_t H_t^* C_t^\prime
	\end{align*}

	The multivariate filter and smoother is then modified by computing several quantities via the univariate filter  on the transformed system: 
	\begin{align*}
	a_{t, i+1} = a_{t,i} + K_{t,i} v_{t,i} &\qquad P_{t,i+1} = P_{t,i} - K_{t,i} F_{t,i} K_{t,i}^{\prime} \\
	v_{t,i} = y_{t,i} - Z_{t,i} a_{t,i} - d_{t,i} \qquad F_{t,i} &= Z_{t,i} P_{t,i} Z_{t,i}^\prime + H_{t,i} \qquad K_{t,i} = P_{t,i} Z_{t,i}^\prime F_{t,i}^{-1} \\ 
	a_{t+1,1} = T_{t+1} a_{t, p+1} + c_{t+1} &\qquad P_{t+1,1} = T_{t+1} P_{t,p+1} T_{t+1}^\prime + R_{t+1} Q_{t+1} R_{t+1}^\prime
	\end{align*}
	where $y_{i,t}$ is the $i$th element of $y_t^*$, $Z_{t,i}$ is the $i$th row of $Z_t^*$, $d_{t,i}$ is the $i$th element of $d_t^*$, and $H_{t,i}$ is the $i$th diagonal element of $H_t^*$. The filtered estimates of the state $a_t = a_{t,1}$ and $P_t = P_{t,1}$ are equivalent to those computed above. Note that $F_{t,i}$ is a scalar, so that $F_{t,i}^{-1}$ is simply scalar division instead of more computationally expensive matrix inversion. However, this comes at the cost of there being no convenient transformation of the quantities $v_{t,i}$, $F_{t,i}$ or $K_{t,i}$ that recovers the multivariate versions computed above.

	Similarly, the univariate smoother provides $r_t = r_{t,0}$ and $N_t = N_{t,0}$:
	\begin{align*}
	r_{t,i-1} = Z_{t,i}^\prime F_{t,i}^{-1} v_{t,i} + L_{t,i}^{\prime} r_{t,i} \qquad N_{t,i-1} =& Z_{t,i}^\prime F_{t,i}^{-1} Z_{t,i} + L_{t,i}^{\prime} N_{t,i} L_{t,i} \qquad L_{t,i} = I_m - K_{t,i} Z_{t,i} F_{t,i}^{-1}\\
	r_{t-1,p} = T_{t}^\prime r_{t,0} &\qquad N_{t-1,p} = T_{t-1}^\prime N_{t,0} T_{t-1}
	\end{align*}
	where $r_{t+1,p} = 0$ and $N_{t+1,p} = 0$. 

\subsection*{Exact Initial Kalman Filter}
	When the state $\alpha_t$ is stationary, the initial values $a_1$ and $P_1$ can be computed as the unconditional mean and variance of the state given the system parameters by inverting the transition equation. 

	To handle cases where some states are nonstationary, the state is separated into states with known variance (those that are stationary) and those that are initialized as diffuse (the nonstationary states). Let $\tilde{R}$ be a selection matrix with columns from the identity such that the initial shock $\eta_1$ is applied to the states with known variance. The selection matrix $A$ is composed of the columns of the identity matrix associated with the diffuse states such that taking the limit as $\kappa \rightarrow \infty$ allows the diffuse states to have infinite initial variance where the initial values are given by 
	\begin{align*}
	a_1 = a + \tilde{R} \eta_1 + A \delta \qquad& \eta_1 \sim N(0, \tilde{Q}) \qquad \delta \sim N(0, \kappa I) \\
	P_1 = P_{*,1} + \kappa P_{\infty,1} \qquad& P_{*,1} = \tilde{R} \tilde{Q} \tilde{R}^\prime \qquad P_{\infty,1} = A A^\prime 
	\end{align*}
	The unconditional mean, $a$, and variance, $P_{*,1}$, of the state are computed by inverting the stationary portion of the system. Any elements of $a$ associated with nonstationary or diffuse states are set to 0. Letting $\tilde{T} = \tilde{R}^\prime T_1 \tilde{R}$ and $\tilde{c} = \tilde{R}^\prime c_1$, this is accomplished by
	\begin{align*}
	\tilde{R}^\prime a &= (I_m - \tilde{T})^{-1} \tilde{c} \\
	\vecop(\tilde{R} \tilde{Q} \tilde{R}^\prime) &= (I_{m^2} - \tilde{T} \otimes \tilde{T})^{-1} \vecop(\tilde{R}^\prime R_1 Q_1 R_1^\prime \tilde{R}) 
	\end{align*}
	When all states are stationary, this initialization collapses down to the simple case where $\alpha_1 \sim N(a_1, P_1)$, where $a_1$ and $P_1$ are the unconditional mean and variance of the state, determined by inverting the full system. 

	Given this initialization, the univariate filter recursions must be altered to separate the states with finite and infinite variances (see \cite{dk_book} (\S 5.2) and \cite{dk_fast}): 
	\begin{align*}
	 F_{*,t,i} = Z_{t,i} P_{*,t,i} Z_{t,i}^\prime + H_{t,i} &\qquad F_{\infty,t,i} = Z_{t,i} P_{\infty,t,i} Z_{t,i}^\prime\\
	K_{*,t,i} = P_{*,t,i} Z_{t,i}^\prime &\qquad K_{\infty,t,i} = P_{\infty,t,i} Z_{t,i}^\prime
	\end{align*}
	\begin{equation*}
	a_{t,i+1} = \begin{cases} 
	      a_{t,i} + K_{*,t,i} F_{*,t,i}^{-1} v_{t,i} & F_{\infty,t,i} = 0 \\
	      a_{t,i} + K_{\infty,t,i} F_{\infty,t,i}^{-1} v_{t,i} & F_{\infty,t,i} \neq 0
    \end{cases} 
    \end{equation*}
	\begin{equation*}
	P_{*,t,i+1} = \begin{cases} 
	   P_{*,t,i} - K_{*,t,i} K_{*,t,i}^\prime F_{*,t,i}^{-1} & F_{\infty,t,i} = 0 \\
	   P_{*,t,i} + K_{\infty,t,i} K_{\infty,t,i}^\prime F_{*,t,i} F_{\infty,t,i}^{-2} - (K_{*,t,i} K_{\infty,t,i}^\prime + K_{\infty,t,i} K_{*,t,i}^\prime) F_{\infty,t,i}^{-1} & F_{\infty,t,i} \neq 0
	\end{cases}
	\end{equation*}
	\begin{equation*}
	P_{\infty,t,i+1} = \begin{cases} 
	   P_{\infty,t,i} & F_{\infty,t,i} = 0 \\
	   P_{\infty,t,i} - K_{\infty,t,i} K_{\infty,t,i}^\prime F_{\infty,t,i}^{-1} & F_{\infty,t,i} \neq 0	   
	\end{cases} 
	\end{equation*}
	\begin{align*}
	a_{t+1,1} &= T_{t+1} a_{t, p+1} + c_{t+1} \\
	P_{\infty,t+1,1} = T_{t+1} P_{\infty,t,p+1} T_{t+1}^\prime &\qquad P_{*,t+1,1} = T_{t+1} P_{*,t,p+1} T_{t+1}^\prime + R_{t+1} Q_{t+1} R_{t+1}^\prime 
	\end{align*}
	For any set of system parameters where the state can be identified, there exists some time $d$ such that $F_{\infty,d,i} = 0$ for all $i$. For time $t > d$, the simpler Kalman filter recursion above can be employed. 

	The smoother must be similarly altered so that beginning at $t = d$, the computation of $r_{t,i}$ is expanded to account for the initialization:
	\begin{align*}
	L_{\infty,t,i} = I_m - K_{\infty,t,i} Z_{t,i} F_{\infty,t,i}^{-1} &\qquad
	L_{*,t,i} = I_m - K_{*,t,i} Z_{t,i} F_{*,t,i}^{-1} \\
	L_{t,i}^{(0)} = \left(K_{\infty,t,i} F_{*,t,i} F_{\infty,t,i}^{-1} - K_{*,t,i} \right) Z_{t,i} F_{\infty,t,i}^{-1} &
	\end{align*}
	\begin{equation*}
	r_{t,i-1}^{(0)} = \begin{cases} 
	   Z_{t,i}^\prime F_{*,t,i}^{-1} v_{t,i} + L_{*,t,i}^\prime r_{t,i}^{(0)} & F_{\infty,t,i} = 0 \\
	   L_{\infty,t,i}^\prime r_{t,i}^{(0)} & F_{\infty,t,i} \neq 0	   
	\end{cases} 
	\end{equation*}
	\begin{equation*}
	r_{t,i-1}^{(1)} = \begin{cases} 
	   r_{t,i}^{(1)} & F_{\infty,t,i} = 0 \\
	   Z_{t,i}^\prime F_{\infty,t,i}^{-1} v_{t,i} + L_{t,i}^{(0)\prime} r_{t,i}^{(0)} + L_{\infty,t,i}^\prime r_{t,i}^{(1)} & F_{\infty,t,i} \neq 0	   
	\end{cases} 
	\end{equation*}
	\begin{equation*}
	r_{t-1,p}^{(0)} = T_t^\prime r_{t,0}^{(0)} \qquad r_{t-1,p}^{(1)} = T_t^\prime r_{t,0}^{(1)}  \\
	\end{equation*}
	\begin{equation*}
	\hat{\alpha}_t = a_t + P_{*,t,1} r_{t,0}^{(0)} + P_{\infty,t,1} r_{t,0}^{(1)}
	\end{equation*}
	where $r_{d,p}^{(0)} = r_{d,p}$ and $r_{d,p}^{(1)} = 0$. Note that $L_{\infty,t,i}$ and $L_{t,i}^{(0)}$ only need to be computed when $F_{\infty,t,i} \neq 0$ and $L_{*,t,i}$ only needs to be computed when $F_{\infty,t,i} = 0$. 

	For the smoothed variance of the state, 
	\begin{align*}
	V_t = P_{*,t,1} - P_{*,t,1} N_{t,0}^{(0)} P_{*,t,1} - \left(P_{\infty,t,1} N_{t,0}^{(1)} P_{*,t,1} \right)^\prime &- P_{\infty,t,1} N_{t,0}^{(1)} P_{*,t,1} - P_{\infty,t,1} N_{t,0}^{(2)} P_{\infty,t,1} \\
	N_{t-1,p}^{(0)} = T_t^\prime N_{t,0}^{(0)} T_t \qquad
	N_{t-1,p}^{(1)} = T_t^\prime N_{t,0}^{(1)} T_t &\qquad
	N_{t-1,p}^{(2)} = T_t^\prime N_{t,0}^{(1)} T_t 
	\end{align*}
	where when $F_{\infty,t,i} = 0$, 
	\begin{align*}
	N_{t,i-1}^{(0)} = Z_{t,i}^\prime F_{*,t,i}^{-1} Z_{t,i} + L_{*,t,i}^\prime N_{t,i}^{(0)} L_{*,t,i} &\qquad
	N_{t,i-1}^{(1)} = N_{t,i}^{(1)} &\qquad
	N_{t,i-1}^{(2)} = N_{t,i}^{(2)} 
	\end{align*}
	and when $F_{\infty,t,i} \neq 0$,
	\begin{align*}
	N_{t,i-1}^{(0)} &= L_{\infty,t,i}^\prime N_{t,i}^{(0)} L_{\infty,t,i}\\
	N_{t,i-1}^{(1)} &= Z_{t,i}^{*\prime} F_{\infty,t,i}^{-1} Z_{t,i} + L_{\infty,t,i}^\prime N_{t,i}^{(0)} L_{t,i}^{(0)} + L_{\infty,t,i}^\prime N_{t,i}^{(1)} L_{\infty,t,i}  \\
	N_{t,i-1}^{(2)} &= Z_{t,i}^\prime F_{\infty,t,i}^{-2} Z_{t,i} F_{*,t,i} + L_{t,i}^{(0)\prime} N_{t,i}^{(1)} L_{t,i}^{(0)} + L_{\infty,t,i}^\prime N_{t,i}^{(1)} L_{t,i}^{(0)} + L_{t,i}^{(0)\prime} N_{t,i}^{(1)} L_{\infty,t,i} + L_{\infty,t,i}^\prime N_{t,i}^{(2)} L_{\infty,t,i}
	\end{align*}
	where $N_{d,p}^{(0)} = N_{d,p}$ and $N_{d,p}^{(1)} = N_{d,p}^{(2)} = 0$. 

% \subsubsection*{Other definitions}
% 	\begin{align*}
% 	{\color{\flag} u_t = F_t^{-1} v_t + K_t^\prime r_t} &\qquad 
% 	{\color{\flag} D_t = F_t^{-1} K_t^\prime N_t K_t} \\
% 	\hat{\eta_t} = Q_t R_t^\prime r_t &\qquad \Var(\eta_t | Y_n) = Q_t - Q_t R_t^\prime N_t R_t Q_t \\
% 	J_t = \Cov(\alpha_{t+1}, \alpha_t | Y_n) &= P_t L_t (I_m - N_t P_{t+1})
% 	\end{align*}

\newpage
\section{Likelihood and Gradient Calculation}
	The likelihood of of data $y_1, \dots, y_n$ in the standard filter as shown in \cite{dk_book} (\S 7.2) is given by the prediction error decomposition:
	\begin{equation*}
	\log L(Y_n) = -\frac{np}{2} \log 2\pi - \frac{1}{2} \sum_{t=1}^n \left(\log |F_t| + v_t^\prime F_{t,i}^{-1} v_t \right)
	\end{equation*}
	
	For the univariate filter, the same decomposition works in the univariate context:
	\begin{equation} \label{eq:likelihood_uni}
	\log L(Y_n) = -\frac{np}{2} \log 2\pi - \frac{1}{2} \sum_{t=1}^n \sum_{i=1}^p\log F_{t,i}+ v_{t,i}^2 / F_{t,i} 
	\end{equation}
	
	The likelihood for the exact initial filter allows for some simplifications in $F_{\infty,t,i}$ as shown in \cite{dk_book} (\S 7.2):
	\begin{equation} \label{eq:likelihood_uni_diff}
	\log L_d(Y_n) = -\frac{1}{2} \sum_{t=1}^n \sum_{i=1}^p \iota_{t,i} \log 2\pi  - \frac{1}{2}  \sum_{t=1}^d \sum_{i=1}^p w_{t,i} - \frac{1}{2} \sum_{t=d}^n \sum_{i=1}^p \iota_{t,i} \left(\log F_{t,i}+ v_{t,i}^2 / F_{t,i} \right)
	\end{equation}
	where  $\iota_{t,i} = 1$ if $F_{*,t,i} \neq 0$ or $ t > d$, and 
	\begin{equation*}
	w_{t,i} = \begin{cases} 
	   \iota_{t,i} (\log (F_{*,t,i}) + v_{t,i}^{(0)2} / F_{*,t,i}) & F_{\infty,t,i} = 0 \\
	   \log F_{\infty,t,i} & F_{\infty,t,i} \neq 0
	\end{cases} 
	\end{equation*}
	Since these quantities are naturally produced by the Kalman filter, this is the preferred method to calculate the likelihood.

\subsection*{Univariate Gradient}
	Following the notation of \cite{nagakura}, we denote $\partial \left[ \vecop(A)^\prime \right] / \partial \theta$ by $\Gt(A)$, where for an $m \times n$ matrix $A$ and a $\theta$ of length $n_\theta$, $\Gt(A)$ will be a $n_\theta \times mn$ matrix of partial derivatives. See Appendix \ref{sec:gradient_derivation} for details and properties of $\Gt(\cdot)$. For the calculation of the gradients of the state space parameters, see section \ref{sec:parameter_estimation}.

	% The gradient of Equation \ref{eq:likelihood_uni} with respect to $\theta$ is then given by: 
	% \begin{equation} \label{eq:gradient_uni}
	% \Gt(\log L(Y_n)) = -\frac{1}{2} \sum_{t=1}^{n} \sum_{i=1}^{p} \Gt(F_{t,i}) \left[F_{t,i}^{-1} - v_{t,i}^2 F_{t.i}^{-2}\right] + 2 \Gt(v_{t,i}) F_{t,i}^{-1} v_{t,i} 
	% \end{equation}
	% where 
	% \begin{align*}
	% \Gt(v_{t,i}) &= - \Gt(Z_{t,i}) a_{t,i} - \Gt(a_{t,i}) Z_{t,i} - \Gt(d_{t,i}) \\ 
	% \Gt(F_{t,i}) &= \Gt(Z_{t,i}) (P_{t,i} Z_{t,i}^\prime) N_m + \Gt(P_{t,i}) (Z_{t,i}^\prime \otimes Z_{t,i}^\prime) + \Gt(H_{t,i})
	% \end{align*}
	% where $\Gt(a_{t,i})$ and $\Gt(P_{t,i})$ are given by the recursion 
	% \begin{align*}
	% \Gt(a_{t,i+1}) &= \Gt(a_{t,i}) + \Gt(K_{t,i}) (v_{t,i} \otimes I_m) + \Gt(v_{t,i}) K_{t,i}^\prime \\
	% \Gt(a_{t+1,1}) &= \Gt(T_{t+1})(a_{t,p+1} \otimes I_m) + \Gt(a_{t,p+1}) T_{t+1}^\prime + \Gt(c_{t+1}) \\
	% \Gt(P_{t,i+1}) &= \Gt(P_{t,i}) - \Gt(K_{t,i})(F_{t,i} K_{t,i}^\prime \otimes I_m) - \Gt(F_{t,i}) (K_{t,i}^\prime \otimes K_{t,i}^\prime) \\
	% \Gt(P_{t+1,1}) &= \Gt(T_{t+1}) (P_{t,p+1} T_{t+1}^\prime \otimes) N_m + \Gt(P_{t,p+1}) (T_{t+1}^\prime \otimes T_{t+1}^\prime) \\
	% &\quad + \Gt(R_{t+1}) (Q_{t+1} R_{t+1}^\prime \otimes I_m) N_m + \Gt(Q_{t+1})(R_{t+1}^\prime \otimes R_{t+1}^\prime) \\
	% \Gt(K_{t,i}) &= \left[ \Gt(P_{t,i}) (Z_{t,i}^\prime \otimes I_m) + \Gt(Z_{t,i}) K_{1m} P_{t,i} \right] F_{t,i}^{-1} - \Gt(F_{t,i}) F_{t,i}^{-2} P_{t,i} Z_{t,i}
	% \end{align*}
	% where $\Gt(a_{0,1}) = \Gt(a_0)$ and $\Gt(P_{0,1}) = \Gt(P_0)$, which are given below. See Appendix \ref{sec:gradient_derivation} for details. 

	Given the univariate likelihood function \eqref{eq:likelihood_uni}, the identities for the gradient in Appendix \ref{sec:gradient_derivation} produce the expression for the univariate gradient: 
	\begin{align*}
	\Gt(\log L(Y_n)) &= \Gt \left(-\frac{np}{2} \log 2\pi - \frac{1}{2} \sum_{t=1}^n \sum_{i=1}^p \log F_{t,i}+ v_{t,i}^2 F_{t,i}^{-1}\right) \\
	&= -\frac{1}{2} \sum_{t=1}^n \sum_{i=1}^p \Gt(\log F_{t,i}) + \Gt(v_{t,i}^2 F_{t,i}^{-1}) \\
	&= - \sum_{t=1}^n \sum_{i=1}^p \frac{1}{2}  \Gt(F_{t,i}) \left[F_{t,i}^{-1} - F_{t,i}^{-2} v_{t,i}^2\right] +  \Gt(v_{t,i}) F_{t,i}^{-1} v_{t,i}
	\end{align*}
	The quantities used in the gradient recursion are computed similarly using the definitions from the univariate filter: 
	\begin{align*}
	\Gt(v_{t,i}) &= \Gt(y_{t,i} - Z_{t,i} a_{t,i} - d_{t,i}) \\
	&= \Gt(y_{t,i}) -\Gt(Z_{t,i}) a_{t,i} - \Gt(a_{t,i}) Z_{t,i}^\prime - \Gt(d_{t,i}) \\
	\Gt(F_{t,i}) &= \Gt(Z_{t,i} P_{t,i} Z_{t,i}^\prime + H_{t,i}) \\
	&= 2 \Gt(Z_{t,i})P_{t,i} Z_{t,i}^\prime + \Gt(P_{t,i}) (Z_{t,i}^\prime \otimes Z_{t,i}^\prime) + \Gt(H_{t,i}) \\
	\Gt(K_{t,i}) &= \Gt(P_{t,i} Z_{t,i}^\prime F_{t,i}^{-1}) \\
	&= \Gt(P_{t,i} Z_{t,i}^\prime) F_{t,i}^{-1} + \Gt(F_{t,i}^{-1}) Z_{t,i} P_{t,i} \\
	&= \Gt(P_{t,i}) (Z_{t,i}^\prime F_{t,i}^{-1} \otimes I_m) + \Gt(Z_{t,i}) P_{t,i} F_{t,i}^{-1} - \Gt(F_{t,i}) F_{t,i}^{-2} Z_{t,i} P_{t,i} \\
	\Gt(a_{t+1,1}) &= \Gt(T_{t+1} a_{t,p+1} + c_{t+1}) \\
	&= \Gt(T_{t+1}) (a_{t,p+1} \otimes I_m) + \Gt(a_{t,p+1}) T_{t+1}^\prime + \Gt(c_{t+1})\\ 
	\Gt(a_{t,i+1}) &= \Gt(a_{t,i} + K_{t,i} v_{t,i}) \\	
	&= \Gt(a_{t,i}) + \Gt(K_{t,i})v_{t,i} + \Gt(v_{t,i})K_{t,i}^\prime \\
	\Gt(P_{t+1,1}) &= \Gt(T_{t+1} P_{t,p+1} T_{t+1}^\prime + R_{t+1} Q_{t+1} R_{t+1}^\prime) \\
	&= \Gt(T_{t+1})(P_{t,p+1} T_{t+1}^\prime \otimes I_m) N_m + \Gt(P_{t,p+1})(T_{t+1}^\prime \otimes T_{t+1}^\prime) \\
	&\quad + \Gt(R_{t+1})(Q_{t+1} R_{t+1}^\prime \otimes I_m) N_m + \Gt(Q_{t+1})(R_{t+1}^\prime \otimes R_{t+1}^\prime) \\
	\Gt(P_{t,i+1}) &= \Gt(P_{t,i} - K_{t,i} F_{t,i} K_{t,i}^\prime) \\ 
	&= \Gt(P_{t,i}) - \Gt(K_{t,i})(F_{t,i}K_{t,i}^\prime \otimes I_m) N_m - \Gt(F_{t,i})(K_{t,i}^\prime \otimes K_{t,i}^\prime)
	\end{align*}
	where $N_m = I_{m^2} + K_{mm}$ for a commutation matrix $K_{mm}$. The initial period gradients $\Gt(a_{1,1}) = \Gt(a_1)$ and $\Gt(P_{1,1}) = \Gt(P_1)$ will be given below. The values of $\Gt(y_{t,i})$ are computed according to 
	\begin{align*}
	\Gt(y_t^*) &= \Gt(C_t^{-1} y_t) \\
	&= -\Gt(C_t) (C_t^{-1} \otimes C_t^{\prime-1}) (y_t \otimes I_p)
	\end{align*}
	where $\Gt(y_{t,i})$ is the $i$th column of $\Gt(y_t^*)$.

\subsection*{Exact Initial Gradient}
	The process for computing the gradient for use with the exact initial filter is similar. Throughout, it is assumed that the boolean outcome of the tests $F_{*,t,i} = 0$ or $F_{\infty,t,i} = 0$ are unaffected by changes to $\theta$. Beginning with the likelihood function, 	
	\begin{align*}
	\Gt(\log L_d(Y_n)) &= \Gt\left(-\frac{1}{2} \sum_{t=1}^n \sum_{i=1}^p \iota_{t,i} \log 2\pi  - \frac{1}{2}  \sum_{t=1}^d \sum_{i=1}^p w_{t,i} - \frac{1}{2} \sum_{t=d}^n \sum_{i=1}^p \iota_{t,i} \left(\log F_{t,i}+ v_{t,i}^2 / F_{t,i} \right) \right)\\
	&= - \frac{1}{2}  \sum_{t=1}^d \sum_{i=1}^p \Gt(w_{t,i}) -\frac{1}{2} \sum_{t=d}^n \sum_{i=1}^p \Gt(F_{t,i}) \left[F_{t,i}^{-1} - F_{t,i}^{-2} v_{t,i}^2\right] + 2 \Gt(v_{t,i}) F_{t,i}^{-1} v_{t,i}
	\end{align*}
	where 
	\begin{align*}
	\Gt(w_{t,i}) &= \begin{cases} 
	   \Gt(\iota_{t,i} (\log (F_{*,t,i}) + v_{t,i}^{(0)2} / F_{*,t,i})) & F_{\infty,t,i} = 0 \\
	   \Gt(\log F_{\infty,t,i}) & F_{\infty,t,i} \neq 0
	\end{cases} \\
	 &= \begin{cases} 
	   \iota_{t,i} \Gt(F_{*,t,i}) \left[F_{*,t,i}^{-1} - F_{*,t,i}^{-2} v_{t,i}^{(0)2}\right] +  \iota_{t,i} 2 \Gt(v_{t,i}^{(0)}) F_{*,t,i}^{-1} v_{t,i}^{(0)} & F_{\infty,t,i} = 0 \\
	   \Gt(F_{\infty,t,i}) F_{\infty,t,i}^{-1}  & F_{\infty,t,i} \neq 0
	\end{cases} 
	\end{align*}

	Again, the required quantities in the recursion come from the definition of the filter: 
	\begin{align*}
	\Gt(F_{*,t,i}) &= \Gt(Z_{t,i} P_{*,t,i} Z_{t,i}^\prime + H_{t,i}) \\
	&= \Gt(Z_{t,i})(P_{*,t,i} Z_{t,i}^\prime) N_m + \Gt(P_{*,t,i}) (Z_{t,i}^\prime \otimes Z_{t,i}^\prime) + \Gt(H_{t,i}) \\
	\Gt(F_{\infty,t,i}) &= \Gt(Z_{t,i} P_{\infty,t,i} Z_{t,i}^\prime + H_{t,i}) \\
	&= \Gt(Z_{t,i})(P_{\infty,t,i} Z_{t,i}^\prime) N_m + \Gt(P_{\infty,t,i}) (Z_{t,i}^\prime \otimes Z_{t,i}^\prime) + \Gt(H_{t,i}) \\
	\Gt(K_{*,t,i}) &= \Gt(P_{*,t,i} Z_{t,i}^\prime) \\
	&= \Gt(P_{*,t,i}) (Z_{t,i}^\prime \otimes I_m) + \Gt(Z_{t,i}) K_{m1} P_{*,t,i} \\
	\Gt(K_{\infty,t,i}) &= \Gt(P_{\infty,t,i} Z_{t,i}^\prime) \\
	&= \Gt(P_{\infty,t,i}) (Z_{t,i}^\prime \otimes I_m) + \Gt(Z_{t,i}) K_{m1} P_{\infty,t,i} 
	\end{align*}

	Between periods, 
	\begin{align*}
	\Gt(a_{t+1,1}) &= \Gt(T_{t+1} a_{t,p+1} + c_{t+1}) \\
	&= \Gt(T_{t+1}) (a_{t,p+1} \otimes I_m) + \Gt(a_{t,p+1}) T_{t+1}^\prime + \Gt(c_{t+1})\\ 
	\Gt(P_{*,t+1,1}) &= \Gt(T_{t+1} P_{*,t,p+1} T_{t+1}^\prime + R_{t+1} Q_{t+1} R_{t+1}^\prime) \\
	&= \Gt(T_{t+1})(P_{*,t,p+1} T_{t+1}^\prime \otimes I_m) N_m + \Gt(P_{*,t,p+1})(T_{t+1}^\prime \otimes T_{t+1}^\prime) \\
	&\quad + \Gt(R_{t+1})(Q_{t+1} R_{t+1}^\prime \otimes I_m) N_m + \Gt(Q_{t+1})(R_{t+1}^\prime \otimes R_{t+1}^\prime) \\
	\Gt(P_{\infty,t+1,1}) &= \Gt(T_{t+1} P_{\infty,t,p+1} T_{t+1}^\prime) \\
	&= \Gt(T_{t+1})(P_{\infty,t,p+1} T_{t+1}^\prime \otimes I_m) N_m + \Gt(P_{\infty,t,p+1})(T_{t+1}^\prime \otimes T_{t+1}^\prime)
	\end{align*}

	When $F_{\infty,t,i} = 0$, 
	\begin{align*}
	\Gt(a_{t,i+1}) &= \Gt(a_{t,i} + K_{*,t,i} F_{*,t,i}^{-1} v_{t,i}) \\
	&= \Gt(a_{t,i}) + \Gt(K_{*,t,i})(F_{*,t,i}^{-1} v_{t,i} \otimes I_m) + \Gt(F_{*,t,i}^{-1} v_{t,i}) K_{*,t,i}  \\
	&= \Gt(a_{t,i}) + \Gt(K_{*,t,i})(F_{*,t,i}^{-1} v_{t,i} \otimes I_m) - \Gt(F_{*,t,i}) F_{*,t,i}^{-2} K_{*,t,i} + \Gt(v_{t,i}) F_{*,t,i}^{-1} K_{*,t,i}  \\
	\Gt(P_{*,t,i}) &= \Gt(P_{*,t,i} - K_{*,t,i} K_{*,t,i}^\prime F_{*,t,i}^{-1}) \\
	&= \Gt(P_{*,t,i}) - \Gt(K_{*,t,i})(F_{*,t,i}K_{*,t,i}^\prime \otimes I_m) - \Gt(F_{*,t,i})(K_{*,t,i}^\prime \otimes K_{*,t,i}^\prime) \\
	\Gt(P_{\infty,t,i+1}) &= \Gt(P_{\infty,t,i}) 
	\end{align*}
	and when $F_{\infty,t,i} \neq 0$, 
	\begin{align*}
	\Gt(a_{t,i+1}) &= \Gt(a_{t,i} + K_{\infty,t,i} F_{\infty,t,i}^{-1} v_{t,i}) \\
	&= \Gt(a_{t,i}) + \Gt(K_{\infty,t,i})(F_{\infty,t,i}^{-1} v_{t,i} \otimes I_m) - \Gt(F_{\infty,t,i}) F_{\infty,t,i}^{-2} K_{\infty,t,i} + \Gt(v_{t,i}) F_{\infty,t,i}^{-1} K_{\infty,t,i}  \\
	\Gt(P_{*,t,i+1}) &= \Gt \left(P_{*,t,i} - K_{\infty,t,i} K_{\infty,t,i}^\prime F_{*,t,i} F_{\infty,t,i}^{-2} - \left(K_{*,t,i} K_{\infty,t,i}^\prime + K_{\infty,t,i} K_{*,t,i}^\prime \right)F_{\infty,t,i}^{-1} \right) \\
	&= \Gt(P_{*,t,i}) - \Gt \left(K_{\infty,t,i} K_{\infty,t,i}^\prime F_{*,t,i} F_{\infty,t,i}^{-2}\right) - \Gt\left((K_{*,t,i} K_{\infty,t,i}^\prime + K_{\infty,t,i} K_{*,t,i}^\prime ) F_{\infty,t,i}^{-1} \right) \\
	&= \Gt(P_{*,t,i}) - \Gt (K_{\infty,t,i}) (K_{\infty,t,i}^\prime \otimes I_m) (F_{*,t,i} F_{\infty,t,i}^{-2} \otimes I_m) \\
		&\quad - \left[ \Gt(F_{*,t,i}) F_{\infty,t,i}^{-2} - 2 \Gt(F_{\infty,t,i}) F_{\infty,t,i}^{-3} F_{*,t,i}) \right] K_{\infty,t,i} K_{\infty,t,i}^\prime \\
		&\quad - [ \Gt(K_{*,t,i}) \left[(K_{\infty,t,i}^\prime \otimes I_m)  + K_{1m} K_{\infty,t,i}^\prime \right] \\ 
		&\qquad + \Gt(K_{\infty,t,i})  \left[ K_{1m} K_{*,t,i}^\prime +  (K_{*,t,i}^\prime \otimes I_m) \right] ] (F_{\infty,t,i}^{-1}  \otimes I_m) \\
		&\quad - \Gt(F_{\infty,t,i}) F_{\infty,t,i}^{-2} (K_{*,t,i} K_{\infty,t,i}^\prime + K_{\infty,t,i} K_{*,t,i}^\prime ) \\
	\Gt(P_{\infty,t,i+1}) &= \Gt(P_{\infty,t,i} - K_{\infty,t,i} K_{\infty,t,i}^\prime F_{\infty,t,i}^{-1}) \\
	&= \Gt(P_{\infty,t,i}) - \Gt(K_{\infty,t,i} K_{\infty,t,i}^\prime) (F_{\infty,t,i}^{-1} \otimes I_m) - \Gt(F_{\infty,t,i}^{-1}) (K_{\infty,t,i} K_{\infty,t,i}^\prime) \\
	&= \Gt(P_{\infty,t,i}) - \Gt(K_{\infty,t,i})(K_{\infty,t,i}^\prime \otimes I_m) N_m (F_{\infty,t,i}^{-1} \otimes I_m) + \Gt(F_{\infty,t,i}) F_{\infty,t,i}^{-2} (K_{\infty,t,i} K_{\infty,t,i}^\prime) \\
	\end{align*}
	since
	\begin{align*}
	\Gt \left(K_{\infty,t,i} K_{\infty,t,i}^\prime F_{*,t,i} F_{\infty,t,i}^{-2}\right) &= \Gt (K_{\infty,t,i} K_{\infty,t,i}^\prime) (F_{*,t,i} F_{\infty,t,i}^{-2} \otimes I_m) \\
	&\quad + \Gt(F_{*,t,i} F_{\infty,t,i}^{-2}) K_{\infty,t,i} K_{\infty,t,i}^\prime \\
	&= \Gt (K_{\infty,t,i}) (K_{\infty,t,i}^\prime \otimes I_m) (F_{*,t,i} F_{\infty,t,i}^{-2} \otimes I_m) \\
	&\quad + \left[ \Gt(F_{*,t,i}) F_{\infty,t,i}^{-2} - 2 \Gt(F_{\infty,t,i}) F_{\infty,t,i}^{-3} F_{*,t,i}) \right] K_{\infty,t,i} K_{\infty,t,i}^\prime \\
	\Gt \left((K_{*,t,i} K_{\infty,t,i}^\prime + K_{\infty,t,i} K_{*,t,i}^\prime ) F_{\infty,t,i}^{-1} \right) &= \Gt \left((K_{*,t,i} K_{\infty,t,i}^\prime + K_{\infty,t,i} K_{*,t,i}^\prime ) \right) (F_{\infty,t,i}^{-1}  \otimes I_m) \\
	&\quad + \Gt(F_{\infty,t,i}^{-1}) (K_{*,t,i} K_{\infty,t,i}^\prime + K_{\infty,t,i} K_{*,t,i}^\prime ) \\
	&= [ \Gt(K_{*,t,i}) (K_{\infty,t,i}^\prime \otimes I_m) + \Gt(K_{\infty,t,i}) K_{1m} K_{*,t,i}^\prime  \\
	&\quad + \Gt (K_{\infty,t,i}) (K_{*,t,i}^\prime \otimes I_m) + \Gt (K_{*,t,i}) K_{1m} K_{\infty,t,i}^\prime  ] (F_{\infty,t,i}^{-1}  \otimes I_m) \\
	&\quad + \Gt(F_{\infty,t,i}) F_{\infty,t,i}^{-2} (K_{*,t,i} K_{\infty,t,i}^\prime + K_{\infty,t,i} K_{*,t,i}^\prime ) \\
	&= [ \Gt(K_{*,t,i}) \left[(K_{\infty,t,i}^\prime \otimes I_m)  + K_{1m} K_{\infty,t,i}^\prime \right] \\ 
	&\quad + \Gt(K_{\infty,t,i})  \left[ K_{1m} K_{*,t,i}^\prime +  (K_{*,t,i}^\prime \otimes I_m) \right] ] (F_{\infty,t,i}^{-1}  \otimes I_m) \\
	&\quad + \Gt(F_{\infty,t,i}) F_{\infty,t,i}^{-2} (K_{*,t,i} K_{\infty,t,i}^\prime + K_{\infty,t,i} K_{*,t,i}^\prime )
	\end{align*}

\subsection*{Initial Conditions}
	The initial conditions for the recursion are given via the expressions for $a_1$ and $P_1$. We assume that the rank of $A$ and $R_0$ are not affected by $\theta$: 
	\begin{align*}
	\Gt(a_1) &= \Gt(a) + \Gt(R_0 \eta_0) + \Gt(A \delta) = \Gt(a) \\
	\Gt(P_1) &= \Gt(\tilde{R} \tilde{Q} \tilde{R}) + \Gt(\kappa A A^\prime) = \Gt(\tilde{R} \tilde{Q} \tilde{R})\\
	% &= \Gt(\tilde{Q})(\tilde{R}^\prime \otimes \tilde{R}^\prime)
	\end{align*}

	From the definitions of $a$ and $\tilde{R} \tilde{Q} \tilde{R}$ above, 
	\begin{align*}
	\Gt(a) &= \Gt([I_m - T_1]^{-1} c_1) \\ 
		&= \Gt([I_m - T_1]^{-1}) (c_1 \otimes I_m) + \Gt(c_1) (I_m - T_1)^{-1})^\prime \\ 
		&= -\Gt(I_m - T_1) [(I_m - T_1)^{-1} \otimes (I_m - T_1)^{\prime-1} ](c_1 \otimes I_m) +  \Gt(c_1) (I_m - T_1)^{\prime-1} \\ 
		&= \Gt(T_1) [(I_m - T_1)^{-1} \otimes (I_m - T_1)^{\prime-1}] (c_1 \otimes I_m) +  \Gt(c_1) (I_m - T_1)^{\prime-1} \\
	\Gt(\tilde{R} \tilde{Q} \tilde{R}^\prime) &= \Gt(S \ \vecop(R_1 Q_1 R_1^\prime))\\
		&= \Gt(S) [\vecop(R_1 Q_1 R_1^\prime) \otimes I_{m^2}] + \Gt(\vecop(R_1 Q_1 R_1^\prime))(I_1 \otimes S^\prime) \\
		&= -\Gt(I_{m^2} - T_1 \otimes T_1)(S \otimes S^\prime) [\vecop(R_1 Q_1 R_1^\prime) \otimes I_{m^2}]\\ 
		&\quad + [\Gt(R_1)(Q_1 R_1^\prime \otimes I_m) N_m + \Gt(Q_1)(R_1^\prime \otimes R_1^\prime)] S^\prime \\
		&= \Gt(T_1 \otimes T_1)(S \otimes S^\prime) [\vecop(R_1 Q_1 R_1^\prime) \otimes I_{m^2}] \\
		&\quad + [\Gt(R_1)(Q_1 R_1^\prime \otimes I_m) N_m + \Gt(Q_1)(R_1^\prime \otimes R_1^\prime)] S^\prime
	\end{align*}
	where $S = (I_{m^2} - T_1 \otimes T_1)^{-1}$. Note that $\Gt(T_1 \otimes T_1)$ must be computed separately. For the diffuse filter, note that $\Gt(P_{*,1}) = \Gt(P_1)$ and $\Gt(P_{\infty,1}) = 0$.

\subsection*{Univariate Treatment of Multivariate Series}

	For the univariate treatment of multivariate series, we need to find $\Gt(Z_t^*)$, $\Gt(d_t^*)$, and $\Gt(H_t^*)$ given $\Gt(Z_t)$, $\Gt(d_t)$, $\Gt(H_t)$, and $C_t$, the univariate factorization matrix. To do so, we  solve for $\Gt(H_t^*)$ and $\Gt(C_t)$, noting that elements of $\Gt(C_t)$ associated with the upper triangular elements of $C_t$ must be zero while elements of $\Gt(H_t^*)$ associated with off-diagonal elements of $H_t^*$ must be zero: 
	\begin{align*}
	\Gt(H_t) &= \Gt(C_t H_t^* C_t^\prime) \\
	&= \Gt(C_t)(H_t^* C_t^\prime \otimes I_p) N_p + \Gt(H_t^*)(C_t^\prime \otimes C_t^\prime) \\
	&= \begin{bmatrix} \Gt(C_t) & \Gt(H_t^*) \end{bmatrix} \begin{bmatrix} (H_t^* C_t^\prime \otimes I_p) N_p \\ C_t^\prime \otimes C_t^\prime \end{bmatrix} 
	= G_t W_t
	= \tilde{G}_t \tilde{W}_t
	\end{align*}
	where $\tilde{G}_t$ contains the nonzero columns of $G_t$ and $\tilde{W}_t$ contains the corresponding rows of $W_t$. The product $\Gt(H_t) \tilde{W}_t^+$ then provides an approximate solution for $\tilde{G}_t$ that can be reordered to solve for $\Gt(C_t)$ and $\Gt(H_t^*)$.

	Having solved for $\Gt(C_t)$, we can solve for the other gradients: 
	\begin{align*} 
	\Gt(Z_t^*) &= -\Gt(C_t)(C_t^{-1} \otimes C_t^{\prime-1})(Z_t \otimes I_p) + \Gt(Z_t)(I_m \otimes C_t^{\prime-1}) \\
	\Gt(d_t^*) &= -\Gt(C_t)(C_t^{-1} \otimes C_t^{\prime-1})(d_t \otimes I_p) + \Gt(d_t)C_t^{\prime-1}
	\end{align*}

\newpage
\section{Parameter Estimation}
	\label{sec:parameter_estimation}
	We are interested in the maximum likelihood estimate of a set of \emph{structural parameters} ($\theta$) given data $y_1, \dots, y_n$ where elements of the \emph{state space parameters} ($Z_t$, $d_t$, $H_t$, $T_t$, $c_t$, $R_t$, and $Q_t$) depend on $\theta$. For computational simplicity, the state space parameters will be restricted such that each of their scalar elements must be a function of a single element of a vector of the \emph{reduced form parameters} ($\psi$) which depend on $\theta$. The reduced form parameters are unrestricted in how they may depend on $\theta$, allowing for a rich specification of parameter restrictions. In most estimations, there will be the same number of structural parameters and reduced form parameters. 

	More formally, let $\theta \in \mathbb{R}^{n_\theta}$ and $\psi \in \mathbb{R}^{n_\psi}$ where $n_\theta \leq n_\psi$. Define a set of functions $\Psi_i: \mathbb{R}^{n_\theta} \rightarrow \mathbb{R}$ such that each element of $\psi$ is a function of the structural parameters, $\psi_i = \Psi_i(\theta)$. Except in cases of cross-parameter restrictions, $n_\theta = n_\psi$ and $\Psi_i(\theta) = \theta_i$ so that $\psi = \theta$. In cases of cross-parameter restrictions, the user must define the $\Psi_i$ functions for each restricted element. Also, unless initial values will be given as the vector $\theta$ (which is often inconvenient), the inverse function $\Psi^{-1} : \mathbb{R}^{n_\psi} \rightarrow \mathbb{R}^{n_\theta}$ must be specified.

	Additionally define a set of functions $\tau_{X_{i,j}}: \mathbb{R} \rightarrow \mathbb{R}$ for $X \in \{ Z_t, d_t, H_t, T_t, c_t, R_t, Q_t \}$ such that each element of the parameter matrices to be estimated is a transformation of an element of $\psi$. Common $\tau_{X_{i,j}}$ transformations include the identity, exponential, negative exponential, and logistic transformations to allow for an unbounded $\psi \in \mathbb{R}^{n_\psi}$ while maintaining bounds on the parameter matrices. When estimating models with mixed-frequency data, all state space parameters shared between the high-frequency and low-frequency states depend on the same reduced form parameters simply by using different $\tau_{X_{i,j}}$ functions. In almost all cases, the specification of the $\tau_{X_{i,j}}$ functions will be done automatically to account for state space parameter bounds and accumulator definitions. User definitions of $\tau_{X_{i,j}}$ should be rare. 

	The gradients of the state space parameters with respect to the reduced form parameters, $G_\psi(X)$, are  easily computed according to the derivatives of $\tau_{X_{i,j}}$. The gradients of the state space parameters with respect to $\theta$ simply require the chain rule, $\Gt(X) = \Gt(\psi) G_\psi(X) $. Note that in cases where $\psi \neq \theta$, $\Gt(\psi)$ must be provided by the user.

	With the likelihood and gradient of the structural parameters defined, the optimization can be performed using common gradient ascent methods available in the Matlab Optimization Toolbox. In practice, it has been observed that further improvements in the likelihood are possible using simplex based optimization methods the gradient ascent method has converged. Due to this, the two methods are repeated until neither is able to improve the likelihood using standard convergence criteria. 

\appendix
\newpage
\section{Gradient Derivation} \label{sec:gradient_derivation}
	Denote $\partial \left[ \vecop(A)^\prime \right] / \partial \theta$ by $\Gt(A)$, where for an $m \times n$ matrix $A$ and a $\theta$ of length $n_\theta$, $\Gt(A)$ will be a $n_\theta \times mn$ matrix of partial derivatives. Several identities from \cite{nagakura} are needed in computing the gradient (assume matrices are of dimensions such that the expression on the left-hand side exists):

	\begin{enumerate}[label=(\alph*)]
	\item \label{gradID:sum} $\Gt(A + B) = \Gt(A) + \Gt(B)$
	\item \label{gradID:mult} $\Gt(A B) = \Gt(A) (B \otimes I_{p_A}) + \Gt(B)(I_{q_B} \otimes A^\prime)$
	\item \label{gradID:trans} $\Gt(A^\prime) = \Gt(A) K_{p_A q_A}$ 
	\item \label{gradID:quad} $\Gt(ACA^\prime) = \Gt(A)(CA^\prime \otimes I_{p_A}) N_{p_A} + \Gt(C)(A^\prime \otimes A^\prime)$
	\item \label{gradID:inv} $\Gt(D^{-1}) = -\Gt(D)(D^{-1} \otimes D^{\prime-1})$ 
	\item \label{gradID:logdet} $\Gt(\log |D|) = \Gt(D) \text{ vec} (D^{\prime-1})$
	\end{enumerate}
	where $C$ is symmetric, $D$ is nonsingular, $K_{mn}$ is a commutation matrix of size $mn$, and $N_m = I_{m^2} + K_{mm}$. For any given matrix $M$ is of dimension $p_M \times q_M$. Note that $\Gt(AA^\prime)$ can easily be computed from \ref{gradID:quad} by treating $C$ as the identity matrix. Also note that $A \otimes I_1 = A$ and $\log |B| = \log B$ if $B$ is a scalar. 

	The chain rule for the operator $\Gt(\cdot)$ is simply $\Gt(A) = \Gt(\psi) G_\psi(A)$ where $A$ is a function of $\psi$ which is a function of $\theta$. This can be shown by noting that 
	\begin{equation*}
	\frac{\partial A_{i,j}}{\partial \theta_k} = \sum_{l = 1}^{n_\psi} \frac{\partial A_{i,j}}{\partial \psi_l} \frac{\partial \psi_l}{\partial \theta_k}
	\end{equation*}
	which defines the appropriate element of the matrix product $\Gt(\psi)G_\psi(A)$.

\bibliography{kalman_filter_smoother} 
\bibliographystyle{unsrt}

\end{document}

